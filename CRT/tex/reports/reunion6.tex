\documentclass[12pt,a4paper]{article} 

\usepackage[defhf,deflayout]{sknife}

%% Common definitions for the project.

% Vincent Hugot

% Does nothing, but does it well
\def\id#1{#1}

% French quotes with proper spacing
\def\quote#1{\flqq\,#1\,\frqq{}}

\def\cad{c'est-�-dire}
\def\Cad{C'est-�-dire}
\def\dateformat#1/#2/#3{#1/#2/#3}

% name formatting
\def\familyname{\textsc}
\def\firstname{\id}
\def\groupmember#1#2{\firstname{#1} \familyname{#2}}

% group members
\def\mwyd{\groupmember{Wydade}{Ben Hamdia}}
\def\mmat{\groupmember{Mathias}{Coqblin}}
\def\mlou{\groupmember{Loubna}{Darkate}}
\def\mmor{\groupmember{Mor Salla}{Fall}}
\def\moli{\groupmember{Olivier}{Finot}}
\def\mvin{\groupmember{Vincent}{Hugot}}
\def\mmed{\groupmember{Mehdi}{Iraqi Houssaini}}
\def\mbed{\groupmember{Bedr}{MOUAJAB}}
\def\mdon{\groupmember{Donald}{Srey}}

% quality of a member during a meeting
\def\quality#1{\ (\textit{#1})}

\def\excused{\quality{excus�}}
\def\absent{\quality{absent}}
\def\animator{\quality{animateur}}
\def\scribe{\quality{scribe}}
\def\reader{\quality{relecteur}}

% theme od a meeting section
\def\meetingtheme#1{ \textbf{[ #1 ]}\sk\\}

\def\mtdebate{\meetingtheme{D�bat}}
\def\mtsuivi{\meetingtheme{Suivi}}
\def\mtplan{\meetingtheme{Planning}}
\def\mtconc{\meetingtheme{Concertation}}

% important points
\def\mtdecision{\textsl}
\def\mtclientsaid{\textsl}

% Questions and answers
\def\qna#1#2{\textbf{#1}\sk\\#2\mk\\}

% technology
\def\tech{\texttt}
\def\java{\tech{Java}}
\def\jjtree{\tech{JJTree}}
\def\javacc{\tech{JavaCC}}
\def\ocaml{\tech{OCaml}}
\def\svn{\tech{SVN}}
\def\junit{\tech{JUnit}}
\def\ant{\tech{ant}}
\def\ast{arbre de syntaxe abstraite}
\def\Ast{Arbre de syntaxe abstraite}
\def\noeud{n\oe{}ud}
\def\eclipse{\tech{Eclipse}}

% some words that it may be useful not to have to type completly each time
\def\op{op�rateur}
\def\ad{addition}
\def\sub{soustraction}
\def\mul{multiplication}
\def\div{division enti�re}
\def\dif{diff�rence}
\def\aff{affectation}
\def\sup{sup�rieur}
\def\inf{inf�rieur}
\def\infe{inf�rieur ou �gal}
\def\dom{d�finition de domaine}

% some defininitions whitch they are not totaly defined yet (and they may not work)

% apparence des nom de classe java non d�finies.
\def\javaclass#1{#1}



% grammar rules
\def\terminal#1{\verb*�#1�}
\def\variable#1{\textbf{#1}}
\def\astnode#1{\textit{\sffamily #1}}

% something the user must replace in a command
\def\userentry#1{$<$\textit{#1}$>$}

% syntax of a console command , for instance
\def\usercommand#1{\texttt{#1}}


% "remets � plus tard ce que tu pourrais faire maintenant !"
\def\todo{{\ }\mk\\\textbf{TODO:}\mk\\}


\pagetitle{CR - Projet CRT \dateformat20/10/2009}
\title{Compte Rendu\\de la Sixi�me R�union\\du Projet de CRT\vspace{-0.5em}}
\author{$\diamond$\vspace{-0.2em}}

\begin{document}
 \maketitle

\section{Personnes Pr�sentes}
%
\noi �taient pr�sents � la r�union :\mk
\begin{itemize}
\item \mwyd \absent
\item \mmat \scribe
\item \mlou
\item \mmor \absent
\item \moli 
\item \mvin 
\item \mmed \absent
\item \mbed \absent
\item \mdon 
\end{itemize}

\section{Bilan g�n�ral}
\subsection{SVN}

Le serveur SVN de la fac nous ayant �t� ``interdit'', le SVN du projet a �t� transf�r� sur \texttt{Google Code} : \texttt{http://code.google.com/p/m2-crt/source/checkout}

\section{Bilan �quipe d�veloppement}
\subsection{Algorithme AC3}
L'algorithme AC3 n'�tait pas totalement au point, il manquait la gestion des substitutions. \mmat\ a termin� l'impl�mentation et l'a test�.

\subsection{Algorithme AC6}
\moli\ a termin� le l'implementation de AC6, dans un �tat similaire � AC3 la r�union d'avant (non-test� et sans gestion des substitutions).

\subsection{Sortie graphique}
\mvin\ a mis en place une sortie \LaTeX\ du graphe de contraintes, et con�u un \texttt{main} pour utiliser l'application (lire un fichier, passer un algorithme, sortie \LaTeX).

\section{Bilan �quipe test}
\hudson\ a �t� abandonn� car incompatible avec le SVN. \fitnesse\ est lui en place, quelques tests ont �t� effectu�s au pr�alable.

%\section{Discussions}


\newpage
\section{Travail � effectuer durant la prochaine it�ration}

Cette it�ration sera la derni�re, le projet sera boucl�. \textbf{Une s�ance de programmation/r�union a �t� fix�e le vendredi 27/11 � 13h30 en 401C.}

\subsection{Developpement}
Il ne reste plus que l'AC6 � terminer, l'ensemble de l'�quipe s'y emploie.

\subsection{Tests}
Reflexions sur les cas de tests, puis mise en place de ces tests avec l'aide de l'�quipe de d�veloppement.


\end{document}

