\SA{Plan de test}

\SB{Introduction}

Lors de ce second projet, qui consiste à appliquer un algorithme de valuation sur le modèles, 3 éléments nouveaux sont apparus :
\begin{itemize}
	\item L'algorithme \fla
	\item Un système de sauvegarde et réstauration de l'état du modèle (pour backtracker)
	\item Une interface graphique
\end{itemize}

\begin{description}
	\item[Le \fla] est évidement l'élément principal à tester.
	Les tests d'acceptation doivent pouvoir couvrir tous les cas particuliers possibles, ainsi que des exemples concrets d'utilisation (fournis ici).
	L'algorithme s'exécute par une simple méthode qui se charge d'appeler toutes ses dépendances; les tests effectués sont donc essentiellement fonctionnels et de type boite noire.
	\item[Le système de sauvegarde et réstauration] ou \emph{backup} et \emph{restore},
	est un système de petite taille, très rapidement implanté, mais crucial pour le bon fonctionnement des méthodes de backtracking.
	Ce système n'ayant pas été mis en place lors du premier projet, il fait l'objet de tests dans ce second projet.
	\item[L'interface graphique] n'a pas beaucoup d'importance ici, elle est simplement mise à disposition pour faciliter le lancement des algorithmes (voir exigences non couvertes en partie \ref{test:notcovered}).
\end{description}


%La réalisation du \fla\ requiert de pouvoir avant chaque décision sauvegarder l'état courant du modèle afin de pouvoir à tout moment revenir en arrière lorsque nécessaire. 
%Or dans l'état actuel des choses, le modèle ne dispose pas de cette fonctionalité, de ce fait il est nécessaire de l'implémenter.


%Lors de l'application de l'algorithme lorsqu'un choix est fait une seule et unique variable est à chaque fois modifiée. C'est pourquoi il a été décidé, à l'origine de créer les deux méthodes suivantes : \sk\\
%\begin{itemize}
%	\item \tech{Variable backup(Variable v)}
%	\item \tech{void restore(Variable v)}
%\end{itemize}



\SB{Périmètre couvert par le plan de test}
\label{test:perimetre}

\SC{Tests unitaires}
Durant la première itération du projet, un plan de test a été conçu pour tester la partie sauvegarde/réstauration.
Il couvre les méthodes \tech{backup} et \tech{restore} du modèle.

\SC{Tests d'intégration}
Les tests d'intégrations avaient déjà été effectués lors du projet précédents, puisqu'il était déjà nécessaire d'enchainer l'utilisation de l'ensemble des modules (parsing, modèle, consistance d'arcs, backtracking).
L'intégration du \fla\ et de l'IHM s'est donc faite de manière informelle.

\SC{Tests fonctionnels}
Les itérations suivant la première se sont concentrées sur les tests d'acceptation pour couvrir l'algorithme \fla.



\SB{Propriétés testées par rapport à l'utilisateur}

Les fonctions \tech{backup} et \tech{restore} permettent de sauvegarder la valeur d'une variable à un moment donné, afin de pouvoir être en mesure de revenir à cet état ultérieurement au besoin.

Les principales choses dont nous devons nous assurer sont :
\begin{itemize}
	\item Que la variable retournée est bien syntaxiquement égale à celle que l'on souhaite sauvegarder,
	\item Que la variable créée est bien une copie de l'originale, c'est à dire que la fonction retourne une référence vers une nouvelle variable et non l'originale. (Sinon toute modification de l'originale sur repercuterait sur la copie qui n'aurait plus aucun intérêt),
	\item Que la variable originale n'est pas modifiée par l'application de la méthode,
	\item Que, lors d'une réstauration, la (ou les) nouvelle valeur dans le modèle est bien prise en compte.
\end{itemize}

Pour l'algorithme \fla, il s'agit simplement de s'assurer que :
\begin{itemize}
	\item Si le modèle, quel qu'il soit, possède une solution, on doit trouver une solution (la première trouvée),
	\item Si le modèle ne possède pas de solution, on doit l'indiquer clairement.
\end{itemize}



\SB{Exigences non couvertes par le plan de test}
\label{test:notcovered}

\begin{description}
	\item[La sauvegarde/réstauration] a été correctement couverte, bien qu'il aurait été bon de tester la bonne neutralité des cas \tech{null} ou de non existance (Variable non reconnue, etc.). Ici on ne peut que confirmer qu'une variable valide a une influence, sinon on ne suppose aucun effet.
	%\item[L'algorithme \fla] a été couvert au mieux possible, avec beaucoup de tests aux limites, de cas particuliers, et les exemples fournis (qui passent avec succès et offrent une réponse correcte).
	\item[L'interface graphique] n'a pas fait l'objet de plan de test, pour les raisons suivantes :
	\begin{itemize}
		\item faible importance du module (il ne s'agit pas d'une application grand publique, mais d'une interface pour faciliter l'utilisation et le test d'un outil technique),
		\item temps de développement très réduit (à cause d'une ``défaillance'' des ressources humaines assignées à la tâche),
		\item les principaux éléments de l'interfaces sont repris d'une ancienne interface (projet MiniJaja de Master 1) déjà beaucoup tester, au mieux quelques tests d'intégrations auraient suffit.
	\end{itemize}
\end{description}


\SB{Stratégies de test}

Deux stratégies principales sont apparues lors de l'application des tests (unitaires comme fonctionnels) :
\begin{itemize}
	\item Une stratégie inspirée du \emph{Pair-wise}, consistant à fixer systématiquement le même système de contraintes, en faisant varier un ou plusieurs éléments pour constater la différence\footnote{Cette stratégie était déjà présente lors du premier projet, par exemple pour tester les erreurs de parsing en introduisant à chaque fois une erreur différente par rapport à un modèle ``sain''.}.
	Pour l'application de cette stratégie, on trouvera généralement un modèle composé de 3 variables ($X$, $Y$ et $Z$), et de deux contraintes.
	Les variables étant également limitées à des intervales simples (\tech{[0..2]}), rendant faciles les tests fonctionnels (dans lesquels il faut pouvoir estimer les résultats possibles).
	
	\item Des \textbf{tests aux limites} sont implantés pour venir compléter les tests effectués.
	Lors des tests de sauvegarde/réstauration, il s'agit d'effectuer plusieurs opérations à la suite sans que celà n'ai d'incidence.
	Lors des tests du \fla, on y trouve des tests sur des grands intervales, des intervales vides, à un seul éléments, des cas sans solutions, etc.
\end{itemize}


\SB{Critères d'acceptation des tests}

Pour passer correctement, les tests de \tech{backup} et \tech{restore} devaient s'assurer de préserver le modèle, et de le restaurer après une modification manuelle.
Pour l'algorithme \fla, une méthode \tech{OneOf} a été implanté pour effectuer les tests :
on propose un modèle de contraintes, on propose l'ensemble des solutions possibles liées à ce modèle, et le test doit confirmer que la solution trouvée en fait partie.


\SB{Critères d'arrêt des tests}

Pour les méthodes \tech{backup} et \tech{restore}, le critère d'arrêt est arrivé assez rapidement étant donné la taille du code : une fois quelques ``enchainements'' de sauvegardes et restaurations effectuées, on peut être assez confiants sur le fonctionnement (voir section \ref{test:risques}).

aucune nouvelle donnée de teste n'a été introduite lorsqu'il a été jugé qu'elles faisaient le travail demandé.

\mk
Pour l'algorithme \fla, nous avons arrêté les tests lorsque plus aucun cas particulier ne s'est présenté.
Pour autant la seule garantie qu'on aie est que les tests sont un succès après un maximum d'expérimentations.


\SB{Gestion des anomalies}

Un bug est apparu dans l'algorithme de consistance AC6 grace aux tests effectués au premier projet.
Ce bug n'a jamais pu être corrigé par manque de temps (les priorités du projet se sont orientées sur le nouvel algorithme à implanté, et une meilleure couverture par les tests).

De même, l'algorithme de backtracking ne semblait pas détecter de solution sur un des exemples fournis (\emph{placement des reines}), et n'avait pas été corrigé.
De plus, lors des tests fonctionnels, un problème critique est apparu lors de la valuation par backtracking (qui ne fonctionnait plus du tout, par un bug insolvable en Java).
Le backtracking a donc été entièrement refondue et simplifiée graces aux nouvelles méthodes de \tech{backup} et \tech{restore}, et fonctionne à nouveau


\SB{Gestion des risques}
\label{test:risques}

Concernant \tech{backup} et \tech{restore}, le code étant rapidement testé, peu de risques étaient pris à leur faire confiance une fois les critères d'arrêt atteinds.
Concernant \fla, puisqu'on n'est jamais sûr du fonctionnement parfait de l'algorithme (étant donné la complexité à mettre en place avec les substitutions), nous avons choisis de considérer comme acceptable les risques pris, faisant corriger l'algorithme au fur et à mesure que l'on trouvait des données de tests invalides.
