\SA{Sauvegarde et restauration du mod�le}

\SB{Introduction}

La r�alisation du \fla\ requiert de pouvoir avant chaque d�cision sauvegarder l'�tat courant du mod�le afin de pouvoir � tout moment revenir en arri�re lorsque n�cessaire. 
Or dans l'�tat actuel des choses, le mod�le ne dispose pas de cette fonctionalit�, de ce fait il est n�cessaire de l'impl�menter.\mk\\

Lors de l'application de l'algorithme lorsqu'un choix est fait une seule et unique variable est � chaque fois modifi�e. C'est pourquoi il a �t� d�cid�, � l'origine de cr�er les deux m�thodes suivantes : \sk\\
\begin{itemize}
	\item \tech{Variable backup(Variable v)}
	\item \tech{void restore(Variable v)}
\end{itemize}



\SB{Pr�rim�tre couvert par le plan de test}

\SB{Propri�t�s test�es par rapport � l'utilisateur}

\SB{Exigences non couvertes par le plan de test}

\SB{Strat�gies de test}

\SB{Crit�res d'acceptation des tests}

\SB{Crit�res d'arr�t des tests}

\SB{Gestion des anomalies}

\SB{Gestion des risques}

\SA{Plan de test \fla}

\SB{Introduction}

\SB{Pr�rim�tre couvert par le plan de test}

\SB{Propri�t�s test�es par rapport � l'utilisateur}

\SB{Exigences non couvertes par le plan de test}

\SB{Strat�gies de test}

\SB{Crit�res d'acceptation des tests}

\SB{Crit�res d'arr�t des tests}

\SB{Gestion des anomalies}

\SB{Gestion des risques}