\SB{Choix d'implémentation}

\SC{Sauvegarde et restauration de Variable}

L’implémentation a commencé avec la mise en place pour la classe modèle des méthodes permettant d'effectuer des sauvegardes de variables.\\
Méthode backup:\\
Syntaxe public Variable backup (Variable v).\\
Cette méthode clone la variable v, son domaine mais garde la référence sur les substitutions et les contraintes.
Par contre pour  une constante qui est une  variable particulière, les références ne sont pas gardées sur les substitutions et les contraintes.
Car, après réflexion on a changé la façon de procéder à la restauration.
En effet, pour sauvegarder et restaurer on a utilisé une HashMap Integer-> Variable, car il y avait un problème: les constantes différentes peuvent avoir le même nom et la même valeur et être "equals".C'est pour cela qu'on a eu besoin de la clé Integer pour déterminer l'emplacement de la constante à restaurer.
On a créé une fonction getKey(Variable) qui regarde les références, au lieu d'utiliser indexOf(qui utilise equals) pour la même raison.\\
Pour restaurer une variable, on fait juste une copie clonée du domaine de la variable sauvegardée, et on l'affecte à la variable du modèle.
En premier lieu, on avait essayé d'affecter un clone de la variable sauvegardée dans la liste des variables, mais ça n'a pas marché, car dans la liste de contraintes et de substitutions les références sur les variables pointaient sur la variable déférencée.

\SC{Valuation}
On garde dans la variable globale \textbf{iindex}, l'indice de la variable qu'on essaye d'instancier.
Pour la valuation , on a dû créer un tableau d'entier qui stocke les valeurs des variables instanciées. Toutes les variables d'indice inférieur strictement à iindex sont instanciées


\SC{Backtracking}
Pour réaliser la partie suivante de l’algorithme : \\
reset each D'k , k>i, to its value before xi was last instanciated .\\
On a crée un tableau de HashMap dont la taille est égale  au nombre de variables.
Au début de la boucle sur la variable i, on sauvegarde les variables de i+1 à N dans HashMap[i].\\

\SC{La méthode de Select value}
 
On a du faire 2 versions du select value, une quand il reste au moins 2 autres variables à instancier si on ne compte pas celle d'indice i qu'on doit instancier, et l'autre quand il en reste 0 ou 1, sans compter i qu'on doit instancier.
     
\SC{Test de consistance}

 Pour tester la consistance entre trois variables  la fonction Consistent a été mis en place. Elle doit rendre le modèle tel qu'elle l'a reçu.
 Les deux versions de  select value, utilisent un AC3 limité à 3 variables, où à l'initialisation et dans le reduce, seuls les arcs ayant une extrémité dans Vi,Vj ou Vk sont révisés.
 Pour le select value avec 2 valeur, on considère on lance la fonction consistent avec k=j.
Pour le select value à 1 valeur, on teste crt.areValidValues, sans passer par la fonction consistent, c'est un test sur la contrainte unaire.
Donc quand on teste la consistance, les contraintes unaires de type A=A ou  A<>A sont prises en compte.

\SC{Substitutions}

On n’ a pas réutilisé les fonctions concernant les substitutions telles qu'elles car elles en faisaient trop. Ainsi on utilise 2 fonctions de substitutions.\\
Tout d'abord, on a la méthode public void updateSubstitutions(int iinitial).\\
Cette méthode met à jour les substitutions de la variable à l'indice iinitial, mais en une seule fois sans faire de révisions reduce. Elle est utilisée par la fonction consistent.\\
Mais avant de lancer l'algorithme AC3 limité à 3 variables de la fonction consistent, on affecte une valeur singleton à toute les variables dont l'indice est inférieur à iindex: l'indice de la variable en cours d'instanciation càd les variables déjà instanciées.
Car on aura besoin de la vrai valeur des variables déjà instanciées pour réutiliser les fonctions sur les substitutions déjà codées.
Ensuite on a utilisé la méthode public void updateSubstitutionsWithoutModifLessEquals(int iinitial) qui pour toutes les valeurs > iindex, met à jour les substitutions pour les variables à partir de iindex+1. Ce qui veut dire que tout ce qui est inférieur ou égal à iindex sera inchangé (restauré ci besoin).
Si on a un reduce, en mettant à jour une des 3 variables impliquées dans la substitution (A=B+C) alors on lance récursivement un
updateSubstitutionsWithoutModifLessEquals sur la variable dont on vient de réduire le domaine.

\SC{Conclusion}
      
Pour l’instant, on ne sait pas si utiliser un AC3 limité à 3 variables au lieu de  AC1 est plus convenable.\\
Lorsqu'on choisit la valeur d'une variable i pour le select value, on dit que les variables artificielles sont des variables, peut être que cela aurait été mieux de n'utiliser que les vrai variables. 
Enfin nous dirons que l’application pourrait comporter quelques bugs vu la complexité de l'algorithme et des substitutions.
