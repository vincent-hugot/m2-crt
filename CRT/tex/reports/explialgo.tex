\SB{Compréhension}
\SC{Propriétés}
Voici pour rappel, les propriétés de l'algorithme de valuation \fla
\begin{itemize}
\item chaque instance est propagée sur les variables restant à instancier
\item réduction des domaines des variables futures
\item lorsqu’un domaine d’une variable future devient vide, l’instanciation courante est rejetée
\item mettent en oeuvre des algorithmes de consistance d’arcs
\end{itemize}
\SC{Décomposition}
L'algorithme \fla\ donné en cours se décompose de la façon suivante
\begin{itemize}
\item une méthode principale qui contrôle le backtracking
\item une méthode select-value qui sélectionne une valeur pour la variable d'indice i à instancier en respectant les propriétés énoncées ci-dessus.
\item une méthode consistent qui en fonction de la variable Vi d'indice i et de valeur a, la variable Vj d'indice j et de valeur b, et la valeur Vk d'indice k, détermine si ces 3 variables sont consistantes entre elles.
\end{itemize}

\SC{Petit exemple}
On considére les trois variables:\\
\\
X de domaine D1=\{1, 2, 3\}\\
Y de domaine D2=\{1, 2, 3\}\\
Z de domaine D3=\{1, 2, 3\}\\
\\
Contraintes :\\
X<Y (1)\\
Y<Z (2)\\
\\
D1’=D1 , D2’=D2 , D3’=D3\\
\\
Déroulement de la fonction select-value pour i=1;\\
a=1 on supprime 1 du domaine de D1’ ainsi D1’=\{2, 3\}\\
j=2 k=3\\
X=1, Y=b, Z=c\\
b=1 inconsistance d’après (1) on supprime 1 de D2’=\{2, 3\}\\
b=2 consistance car on peut choisir c=3 par exemple\\
b=3 inconsistance d’après (2) on supprime 3 de D2’=\{2\}\\
\\
j=3 k=2\\
X=1, Y=c, Z=B\\
b=1 inconsistance on supprime 1 de D3’ ainsi D3’=\{2, 3\} l'inconsistance est détectée après application d'une consistance d'arc (par exemple AC3 réduit à seulement ces 3 variables)\\
b=2 inconsistance on supprime 2 de D3’ ainsi D3’=\{3\} \\
b=3 consistance car on peut choisir c=2 par exemple\\
return a=1\\
D1’=\{2, 3\} , D2’=\{2\} , D3’=\{3\}\\
\\
Déroulement de la fonction select-value pour i=2;\\
return a=2\\
\\
Déroulement de la fonction select-value pour i=3;\\
return a=3\\
D1’=\{2, 3\} , D2’=\{\} , D3’=\{\}\\
i=4\\
On retourne \{X1=1, X2=2, X3=3\}.

\SC{Conclusion du petit exemple}
Quand i=2 et i=3, on ne passe pas par la boucle avec le double for.\\
On va avoir besoin d'une fonction de consistence du type AC3 mais qui se limite à 3 variables.\\
On va avoir besoin de sauvegarder l'état d'une variable(son domaine plus précisément).\\
Il va falloir sauvegarder la valeur des variables instanciées dans un tableau par exemple.
