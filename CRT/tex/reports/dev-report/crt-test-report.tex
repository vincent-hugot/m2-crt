\documentclass[a4paper,12pt]{article}
\usepackage[pstricks]{sk2}

\author{Groupe CRT 1\\Test}

\title{Rapport de projet}

\def\tech{\texttt}

% name formatting
\def\familyname{\textsc}
\def\firstname#1{#1}
\def\groupmember#1#2{\firstname{#1} \familyname{#2}}

% group members
\def\mwyd{\groupmember{Wydade}{Ben Hamdia}}
\def\mmat{\groupmember{Mathias}{Coqblin}}
\def\mlou{\groupmember{Loubna}{Darkate}}
\def\mmor{\groupmember{Mor Salla}{Fall}}
\def\moli{\groupmember{Olivier}{Finot}}
\def\mvin{\groupmember{Vincent}{Hugot}}
\def\mmed{\groupmember{Mehdi}{Iraqi Houssaini}}
\def\mbed{\groupmember{Bedr}{MOUAJAB}}
\def\mdon{\groupmember{Donald}{Srey}}

\def\grpd{Groupe Développement}
\def\grpt{Groupe Test}

\begin{document}
 
\maketitle

\tableofcontents

\section{Réunion initiale (zéro): 2009-09-30}

Durant cette première réunion durant le TP de CRT, nous 
nous sommes répartis en deux sous-groupes, l'un dédié au développement
de l'application, et l'autre au test.\mk\\
%
La répartition nominative de ces groupes est la suivante:
\begin{itemize}
\item \mvin : Chef de projet, chef \grpd
\item \mmat
\item \moli
\item \mbed
\item \mmed
\end{itemize}


\begin{itemize}
\item \mwyd : Chef \grpt
\item \mdon
\item \mlou
\item \mmor
\end{itemize}


\noi Les tâches définies pour l'itération étaient:

\begin{itemize}
 \item \moli\ et \mmat: Création (sur papier pour l'instant) d'un modèle objet cohérent sur lequel fonder le développement.
\item  \mvin : Définition du langage d'entrée et écriture d'un parseur complet pour ce langage.
\item Le \grpt, et d'une manière générale tout personne n'ayant pas de tâche précise,
était chargé de se documenter sur les outils de test
unitaire, plugins \texttt{Eclipse} dédiés aux tests etc
\end{itemize}
La raison pour laquelle tout le monde n'avait pas de tâche précise est que la 
quasi-totalité des tâches de développement dépendent du modèle, lequel n'était 
pas encore défini.\mk\\
%
Ce document étant le rapport du \grpd, les tâches du \grpt\ seront
omises dans les sections suivantes.

\section{Première réunion: 2009-10-6}

\subsection{Bilan de l'itération passée}

\subsection{Tâches pour la prochaine itération}

\begin{itemize}
 \item \mwyd\ et \mlou\ se sont occupé d'obtenir des informations sur un éventuel plugin Cobertura pour Eclipse.
 \item \mdon\ et \mmor\ se sont occupé d'obtenir des informations sur un éventuel plugin Emma pour Eclipse
\end{itemize}



\section{Seconde réunion: 2009-10-14}

\subsection{Bilan de l'itération passée}
\begin{itemize}
 \item Plugin Eclipse pour Cobertura :
 Il n'existe apparemment pas de plugin Eclipse pour Cobertura. Les rapports HTML générés ne donnent aucun résultat(comme en TP).Il doit nous manquer quelque chose...

\item Plugin Eclipse pour EMMA :
 Il existe le plugin Eclemma. Il a été installé et testé sans problème.
\end{itemize}

\subsection{Tâches pour la prochaine itération}

\begin{itemize}
 \item \mlou\ et \mmor\ ont été chargé de mettre en place le serveur de Build continu Hudson.
 \item \mdon\ et \mwyd\ ont été chargé de choisir le plugin à utiliser pour les tests JUnit (Cobertura ou Emma), puis d'aider à mettre en place le serveur Hudson, enfin de se documenter sur Fitnesse.
 \item Pour l'équipe test: Faire passer les tests U: batterie des tests du parsing du fichier.

\end{itemize}

\section{Troisième réunion: 2009-10-21}

\subsection{Bilan de l'itération passée}

\begin{itemize}

\item Aprés une réunion, \mdon\ , \mwyd\ et \mmor\ ont conseillé aux développeurs d'utiliser le plugin Emma après avoir vu qu'il était plus simple d'utilisation.

 \item La majeure partie du travail effectuée par l'équipe test a consisté à essayer de mettre
en place le serveur Hudson malheureusement sans succès. A ce sujet Emilie OUDOT, a
mentionné que le fait que nous n'ayons pas de droits administrateurs sur Hudson puisse
poser des problèmes, elle viendra en TP pour nous aider à résoudre ce soucis.
D'autre part aucun travail n'a été fait sur Fitnesse, en outre il a été décidé d'utiliser
Emma plutôt que Cobertura car certains tests ne passaient pas sous Cobertura et de plus
Hudson intègre un plugin Emma.
Ont été clairement redéfini les rôles des équipes quant aux divers tests à effectuer :

L'équipe de developpement s'occupe des tests structurels, c'est à dire s'assurer que
chaque méthode fait ce qui lui est demandé (tests unitaires avec JUnit).

L'équipe de test s'occupe des tests structurels plus globaux (Fitnesse).
Pour ce qui est de Fitnesse, a été décidé le fonctionnement suivant :
Dans un premier temps, l'équipe de test doit définir ce qui doit être testé, quelles sont les
caractéristiques attendues.

\end{itemize}


\subsection{Tâches pour la prochaine itération}

\begin{itemize}

\item La première chose à faire est la mise en place du serveur Hudson, ce qui devrait être
fait durant la séance de TP.
Ensuite il faudra s'occuper de Fitnesse.\dots

\item Le travail à faire(proposé par l'équipe de dev) est de construire un langage régulier pour effectuer des tests sur Fitnesse.
On s'apercevra par la suite que ce n'était pas nécessaire car Fitnesse ne nécessite pas de langage régulier.
Ensuite, on devait tester si des valeurs appartenaient ou non au domaine d'une variable.

\item On devait aussi s'occuper des tests de parsing du fichier qui étaient à faire pour cette itération.

\end{itemize}

\section{Quatrième réunion: 2009-10-30}

Réunion pendant les vacances, faisant le bilan du travail
effectué pendant les premiers jours de "vacances".

\subsection{Bilan de l'itération passée}

\begin{itemize}
 \item On a suivi des tutorials sur Fitnesse. On a effectué quelques tests de parsing du fichier sur un serveur Fitnesse en local:
 on a vérifié que le modèle renvoyé était nul (erreur synthaxique) ou non nul (pas d'erreur synthaxique).
 \item Le serveur Hudson ne marche toujours pas malgré l'intervention d'Emilie OUDOT(problème de droits administrateurs).
\end{itemize}


\subsection{Tâches pour la prochaine itération}

\`A ce stade, le développement était presque terminé, puisqu'il
 ne restait plus qu'à ajouter le support des substitutions à AC6.\mk\\
%
Les réunions qui ont suivi ont été principalement consacrées à
la discussion avec l'équipe de test sur Fitnesse.

\begin{itemize}
 \item Mettre en place le serveur Fitnesse avec des tests sur AC3, AC6 et le parseur.
\end{itemize}


\section{Dernière réunion/séance de développement: 2009-11-27}

\begin{itemize}

\item Durant cette longue période, nous avons eu un serveur Hudson avec les droits admin, mais en même temps l'équipe réseau a coupé l'accès en SSH à svnlmdinfo.univ-fcomte.fr.
L'équipe de dev a donc migré notre SVN sur Google Code. Toutefois Hudson bloquait les commandes SVN par HTTP, on a donc abandonné Hudson.

 \item Nous avons entre temps mis en place un serveur Fitnesse en local sur le port 1342 accessible via localhost:1342. Il est lancé par la commande ant fitnesse.
 Il y a un lien pour stopper ce serveur depuis la page d'accueil, mais on peut aussi l'arrêter en stoppant le programme fitnesse\.jar.
 Nous avons implémenter les tests du parseurs et des tests sur AC3.

 \item Durant cette réunion, nous avons implémenté les mêmes tests qu'AC3 pour AC6, et l'équipe de dev nous a aidés à simplifier notre code.

\item Nous avons commencé à rédiger un rapport de nos itérations.

\end{itemize}

\section{Remarque générale}

Nos principaux problèmes ont été la mise en place du serveur Hudson qui a été entravé par les nécessités de sécurité de l'Université, et 
aussi la compréhension globale du projet concernant les algorithmes AC3, AC6 et les différentes parties du programme.
Donc finalement, nous nous sommes passés du serveur de build continu. Nous avons également oublié d'effectuer des tests avec l'opérateur de division.

\section{Tests remarquables}

Voici un test remarquable: lors de l'opération y*y, le 2eme y est considéré comme une variable différente du 1er y.
Cette action est normale car les algo AC3 et AC6 ne font qu'effectuer un filtrage, il n'y a pas de valuation.
Nos tests sur AC6 ont fait apparaître que l'algo AC6 éliminait moins de valeurs qu'AC3.

\end{document}
