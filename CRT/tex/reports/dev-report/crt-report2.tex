 
\documentclass[a4paper,12pt]{article}
\usepackage[pstricks]{sk2}
\usepackage{pxfonts}

\author{Groupe CRT 1}

\title{Rapport de projet CRT}

\def\tech{\texttt}

% name formatting
\def\familyname{\textsc}
\def\firstname#1{#1}
\def\groupmember#1#2{\firstname{#1} \familyname{#2}}

% group members
\def\mwyd{\groupmember{Wydade}{Ben Hamdia}}
\def\mmat{\groupmember{Mathias}{Coqblin}}
\def\mlou{\groupmember{Loubna}{Darkate}}
\def\mmor{\groupmember{Mor Salla}{Fall}}
\def\moli{\groupmember{Olivier}{Finot}}
\def\mvin{\groupmember{Vincent}{Hugot}}
\def\mmed{\groupmember{Mehdi}{Iraqi Houssaini}}
\def\mbed{\groupmember{Bedr}{Mouajab}}
\def\mdon{\groupmember{Donald}{Srey}}

\def\grpd{Groupe Développement}
\def\grpt{Groupe Test}

\def\fla{Full Look Ahead}
\def\tdd{Test Driven Development}

\newcommand\seclvlA{
\newcommand\SA\section
\newcommand\SB\subsection
\newcommand\SC\subsubsection}

\newcommand\seclvlB{
\newcommand\SA\subsection
\newcommand\SB\subsubsection
\newcommand\SC\paragraph}

\begin{document}
 
\maketitle

\tableofcontents



\section{Introduction}

Dans cette seconde partie du projet de CRT, notre groupe était
chargé d'implanter l'algorithme de valuation FLA (\texttt{Full Look Ahead}).
Nous avons également pourvu notre programme d'une interface graphique,
en plus du mode \emph{ligne de commande} disponible précédemment,
corrigé certains problèmes qui subsistaient dans
les algorithmes implantés dans la première partie du
projet.

\section{Organisation}

Comme à l'accoutumée, nous nous sommes astreints à respecter la
méthode XP autant que faire se peut étant donné les autres contraintes
auquelles nous étions soumis.\mk\\
%
Nous avons, comme prévu, interchangé les groupes de développement et 
de test par rapport à la première partie du projet. 
Les responsables respectifs des deux groupes ont également changé.
La nouvelle répartition suit:\mk\\
%
\textbf{\grpt}
\begin{itemize}
\item $\frac12$\mvin : Chef de projet
\item \mmat : Chef \grpt
\item \moli
\item \mbed
\item \mmed \mk\\
\end{itemize}

\textbf{\grpd}
\begin{itemize}
\item \mdon : Chef \grpd
\item \mmor
\item \mwyd 
\item \mlou
\item $\frac12$\mvin 
\end{itemize}

\noi Une particularité de la nouvelle répartition est que \mvin\ a
été assigné ``pour moitié'' à chaque groupe, d'une part
afin d'aider la transition de l'ancien \grpt\ vers le développement,
et d'autre part afin d'éviter de répéter les erreurs faites dans la première
partie du travail. En effet nous avions souffert dans la première partie
d'un manque de communication entre les deux équipes.
Avoir le chef de projet consacrer son temps équitablement
entre les deux équipes nous a semblé une solution raisonnable
à ce problème.\mk\\
%
Nous avons travaillé par itérations d'environ une semaine,
chacune ponctuée par une réunion (dont une pendant les vacances)
donnant lieu à un rapport de réunion.
En dehors des réunions, l'outil d'organisation
principal restait notre forum, du moins pour ceux
qui y passent plus d'une fois par an.\mk\\
%
Comme pour la première partie du projet, la répartition
des tâches établie en réunions n'a pas été scrupuleusement
appliquée en pratique, 
en particulier car certains binômes ont
accumulé beaucoup de retard et/ou rendu
un travail inexploitable, qui a dû être refait par 
d'autres membres de l'équipe lorqu'il a été clairement
établi que ce n'était plus la peine d'attendre.


\section{Algorithme \texttt{Full Look Ahead}}


%input{}


\seclvlA
\SA{Sauvegarde et restauration du mod�le}

\SB{Introduction}

La r�alisation du \fla\ requiert de pouvoir avant chaque d�cision sauvegarder l'�tat courant du mod�le afin de pouvoir � tout moment revenir en arri�re lorsque n�cessaire. 
Or dans l'�tat actuel des choses, le mod�le ne dispose pas de cette fonctionalit�, de ce fait il est n�cessaire de l'impl�menter.\mk\\

Lors de l'application de l'algorithme lorsqu'un choix est fait une seule et unique variable est � chaque fois modifi�e. C'est pourquoi il a �t� d�cid�, � l'origine de cr�er les deux m�thodes suivantes : \sk\\
\begin{itemize}
	\item \tech{Variable backup(Variable v)}
	\item \tech{void restore(Variable v)}
\end{itemize}



\SB{Pr�rim�tre couvert par le plan de test}

\SB{Propri�t�s test�es par rapport � l'utilisateur}

La fonction backup permet de sauvegarder la valeur d'une variable � un moment donn�, afin de pouvoir �tre en mesure de revenir � cet �tat ult�rieurement au besoin.

Les pricipales choses dont nous devons nous assurer sont :
\begin{itemize}
\item Que la variable retourn�e est bien syntaxiquement �gale � celle que l'on souhaite sauvegarder
\item Que la variable cr��e est bien une copie de l'originale, c'est � dire que la fonction retourne une r�f�rence vers une nouvelle variable et non l'originale. (Sinon toute modification de l'originale sur repercuterait sur la copie qui n'aurait plus aucun int�r�t)
\item Que la variable originale n'est pas modifi�e par l'application de la m�thode.

\end{itemize}

\SB{Exigences non couvertes par le plan de test}
Il aurait �t� bon de tester que l'application de la m�thode en lui passant en param�tre une variable nulle ou non pr�sente dans le mod�le retourne bien une erreur.
\SB{Strat�gies de test}
 
Tous les tests se d�roulent dans un m�me ``environnement''. C'est � dire qu'ils sont tous effectu�s sur un mod�le contenant 3 variable et deux contraintes.
Un test a �t� �crit pour chacune des exigences cit�es. Aisin qu'un test aux limites en essayant de sauvegarder une nouvelle varaiable ayant un domaine vide. De plus un dernier test permet de s'assurer que la fonction n'alt�re aucune des autres variables du mod�le.
\SB{Crit�res d'acceptation des tests}

\SB{Crit�res d'arr�t des tests}

\SB{Gestion des anomalies}

\SB{Gestion des risques}

\SA{Plan de test \fla}

\SB{Introduction}

\SB{Pr�rim�tre couvert par le plan de test}

\SB{Propri�t�s test�es par rapport � l'utilisateur}

\SB{Exigences non couvertes par le plan de test}

\SB{Strat�gies de test}

\SB{Crit�res d'acceptation des tests}

\SB{Crit�res d'arr�t des tests}

\SB{Gestion des anomalies}

\SB{Gestion des risques}

\section{Bilan technique}

\textbf{Algorithme par algorithme:}
\begin{itemize}
\item L'algorithme AC3 fonctionne bien, mais est relativement lent lorsque les problèmes 
génèrent de très nombreuses substitutions (quelque milliers).

\item L'algorithme AC6 fonctionne, avec les mêmes caveats qu'AC3, mais donne de moins bonnes
réductions d'AC3.

\item Le backtracking fonctionne essentiellement, mais souffre de quelques bugs étranges pour l'instant. 
  \todo{to be continued}

\item Le Full Look Ahead fonctionne bien.
  
\end{itemize}

\textbf{Exemple par exemple:}
\begin{itemize}
 \item \textbf{Dames: } Aucun algorithme de consistance ne réduit les domaines. 
    Cependant le \texttt{FLA} trouve une solution.
 \item \textbf{Zebre: } Tous les algorithmes fonctionnent bien, modulo la remarque faite plus haut en ce qui 
    concerne AC6.
 \item \textbf{SEND MORE = MONEY: } L'exemple se résoud bien, mais pas dans un temps raisonnable
  lorsque tous les domaines sont ouverts. Pour l'exécuter, il nous faut réduire le domaine de quatre
  variables à un singleton (la bonne valeur, évidemment), et l'algorithme trouve les autres rapidement.
  Cette mauvaise performance est dûe à la gestion des substitutions, fort nombreuses sur l'exemple.
\end{itemize}

\section{Conclusion}

En dépit des problèmes d'organisation que nous avons rencontrés, 
le travail rendu respecte essentiellement le cahier des charges,
même s'il reste de nombreux points perfectibles.

\end{document}
