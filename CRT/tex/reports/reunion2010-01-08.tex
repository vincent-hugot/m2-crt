\documentclass[12pt,a4paper]{article} 

\usepackage[defhf,deflayout]{sknife}

%% Common definitions for the project.

% Vincent Hugot

% Does nothing, but does it well
\def\id#1{#1}

% French quotes with proper spacing
\def\quote#1{\flqq\,#1\,\frqq{}}

\def\cad{c'est-�-dire}
\def\Cad{C'est-�-dire}
\def\dateformat#1/#2/#3{#1/#2/#3}

% name formatting
\def\familyname{\textsc}
\def\firstname{\id}
\def\groupmember#1#2{\firstname{#1} \familyname{#2}}

% group members
\def\mwyd{\groupmember{Wydade}{Ben Hamdia}}
\def\mmat{\groupmember{Mathias}{Coqblin}}
\def\mlou{\groupmember{Loubna}{Darkate}}
\def\mmor{\groupmember{Mor Salla}{Fall}}
\def\moli{\groupmember{Olivier}{Finot}}
\def\mvin{\groupmember{Vincent}{Hugot}}
\def\mmed{\groupmember{Mehdi}{Iraqi Houssaini}}
\def\mbed{\groupmember{Bedr}{MOUAJAB}}
\def\mdon{\groupmember{Donald}{Srey}}

% quality of a member during a meeting
\def\quality#1{\ (\textit{#1})}

\def\excused{\quality{excus�}}
\def\absent{\quality{absent}}
\def\animator{\quality{animateur}}
\def\scribe{\quality{scribe}}
\def\reader{\quality{relecteur}}

% theme od a meeting section
\def\meetingtheme#1{ \textbf{[ #1 ]}\sk\\}

\def\mtdebate{\meetingtheme{D�bat}}
\def\mtsuivi{\meetingtheme{Suivi}}
\def\mtplan{\meetingtheme{Planning}}
\def\mtconc{\meetingtheme{Concertation}}

% important points
\def\mtdecision{\textsl}
\def\mtclientsaid{\textsl}

% Questions and answers
\def\qna#1#2{\textbf{#1}\sk\\#2\mk\\}

% technology
\def\tech{\texttt}
\def\java{\tech{Java}}
\def\jjtree{\tech{JJTree}}
\def\javacc{\tech{JavaCC}}
\def\ocaml{\tech{OCaml}}
\def\svn{\tech{SVN}}
\def\junit{\tech{JUnit}}
\def\ant{\tech{ant}}
\def\ast{arbre de syntaxe abstraite}
\def\Ast{Arbre de syntaxe abstraite}
\def\noeud{n\oe{}ud}
\def\eclipse{\tech{Eclipse}}

% some words that it may be useful not to have to type completly each time
\def\op{op�rateur}
\def\ad{addition}
\def\sub{soustraction}
\def\mul{multiplication}
\def\div{division enti�re}
\def\dif{diff�rence}
\def\aff{affectation}
\def\sup{sup�rieur}
\def\inf{inf�rieur}
\def\infe{inf�rieur ou �gal}
\def\dom{d�finition de domaine}

% some defininitions whitch they are not totaly defined yet (and they may not work)

% apparence des nom de classe java non d�finies.
\def\javaclass#1{#1}



% grammar rules
\def\terminal#1{\verb*�#1�}
\def\variable#1{\textbf{#1}}
\def\astnode#1{\textit{\sffamily #1}}

% something the user must replace in a command
\def\userentry#1{$<$\textit{#1}$>$}

% syntax of a console command , for instance
\def\usercommand#1{\texttt{#1}}


% "remets � plus tard ce que tu pourrais faire maintenant !"
\def\todo{{\ }\mk\\\textbf{TODO:}\mk\\}


\pagetitle{CR - Projet CRT \dateformat08/01/2009}
\title{Compte Rendu\\de la R�union\\du Projet de CRT\vspace{-0.5em}}
\author{$\diamond$\vspace{-0.2em}}

\begin{document}
 \maketitle

\section{Personnes Pr�sentes}
%
\noi �taient pr�sents � la r�union :\mk
\begin{itemize}
\item \mwyd 
\item \mmat 
\item \mlou \absent
\item \mmor \absent
\item \moli 
\item \mvin \scribe
\item \mmed 
\item \mbed 
\item \mdon 
\end{itemize}

\section{Bilan g�n�ral}

\section{Bilan �quipe d�veloppement}

\subsection{Algorithme Full Look-Ahead}
\noi L'impl�mentation de l'algorithme Full Look-Ahead ne g�rant pas les substitutions est au point.
Des tests unitaires ont �t� �crits pour tester l'algorithme.

\subsection{IHM}
\noi \mlou\ n'�tait pas pr�sente, et n'a pas avanc�
sur l'IHM car elle ne conna�t pas Java. \mk\\
%
\mwyd\ a implant� une IHM tr�s tr�s rudimentaire, 
pour l'instant non reli�e aux fonctions du projet.

\section{Bilan �quipe test}

\subsection{Plan de test}
\noi \moli\ n'a pas avanc� sur le plan de test.\mk\\
%
\mmat\ a d�fini un plan de test sur papier, qui n'est pas encore r�dig� ni implant� sous
Fitnesse.

\section{Bilan "t�ches diverses"}

\subsection{Reprise des rapports en retard}
\noi \moli\ n'a pas avanc� sur le rapport de \mlou.\mk\\
%
\mbed\ a fait, mais non commit�, un rapport n'utilisant pas les macros \LaTeX\ \emph{ad hoc}.
Ces t�ches sont � compl�ter hors it�ration (pas d'extension de d�lai).

\section{Discussions}

\begin{itemize}
\item Les tests unitaires sur l'algorithme prennent beaucoup trop de temps; ils utilisent des
domaines trop gros et, pour certains, rel�vent davantage du test fonctionnel.
\end{itemize}


\section{Travail � effectuer durant la prochaine it�ration}

\subsection{Developpement}

L'�quipe se divise sur deux t�ches :

\begin{itemize}
\item \mdon\, \mmor\ continueront l'impl�mentation de l'algorithme Full Look-Ahead 
(ajout du support des substitutions) avec l'aide du premier quart de \mvin.
\item \mlou\, \mwyd\ essayeront de d�velopper une IHM moins minimaliste que celle 
disponible actuellement, et de la rendre fonctionnelle. Elles recevront l'aide
des trois derniers quarts de \mvin\ pour la partie ``affichage graphique du mod�le''.
\end{itemize}

\subsection{Tests}

Deux plans de test sont � faire:

\begin{itemize}
\item \mmed, \mbed, $\f12$ \mmat\ travailleront sur le plan de test du FLA avec substitutions.
\item \moli\ et $\f12$ \mmat\ reprendront les tests unitaires du FLA pour r�duire leur temps
d'ex�cution, supprimer les doublons et transformer en tests fonctionnels ceux qui 
ont une port�e trop large pour rester dans les tests unitaires.
\end{itemize}


\end{document}

